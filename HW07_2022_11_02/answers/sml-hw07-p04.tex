\documentclass[12pt]{article}
\usepackage{fancyhdr, amsmath, amssymb, amsthm}
\usepackage[margin=1in]{geometry}


\newcommand*{\QEDA}{\null\nobreak\hfill\ensuremath{\square}}

\setlength{\parindent}{0em}
\setlength{\parskip}{0.1em}


\pagestyle{fancy}
\fancyhf{}

\rhead{Stochastic Methods Lab \\ Name: Ayam Banjade}
\lhead{Jacobs University Bremen \\
	Submitted to: Sören Petrat}

\begin{document}
	\begin{center}
        \Large{\texttt{Homework 7 Problem 4}} \vspace*{0.8mm}  
	\end{center}
    \textbf{Theorem:} \textit{It is never optimal to exercise an American call option on a non-dividend-paying stock before expiration.}\\

    \textbf{Proof} 
    \begin{align*}
        \textbf{Portfolio A: } & \text{American call option and } Ke^{-r(T-t)} \text{ in cash} \\
        \textbf{Portfolio B: } & \text{One share}
    \end{align*}
    Let, $S_t$ be share price at time $t$. \\

    For \textbf{Portfolio A}: assuming the option is exercised at some time $t < T$ 
    \begin{align*}
       \text{value of \textbf{A}} &= ( S_t - K ) + Ke^{-r(T-t)} < S_t \\
       \text{value of \textbf{B}} &= S_t
    \end{align*}
    Now, assuming the option is exercised at $T$
    \begin{align*}
       \text{value of \textbf{A}} &= max( S_T - K, 0 ) + K  \\
                                  &= max(S_T, K) \geq S_T \\
       \text{value of \textbf{B}} &= S_T
    \end{align*}
    So, exercising the option before maturity gives a portfolio with value less than that of \textbf{Portfolio B}. If exercised at time of maturity - the value is greater than or equal to the value of \textbf{Portfolio B}. So in the given case, an American call option should not be exercised early.
    \QEDA

    \clearpage

\end{document}
